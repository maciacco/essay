
%%%%%%%%%%%%%%%%%%%%%%%%%%%%%%%%%%%%%%%%%%%%%%%%%%%%%%%%%%%%%%%%%%
%%                 DOCUMENT FOR QFT 0 ESSAY                     %%
%%%%%%%%%%%%%%%%%%%%%%%%%%%%%%%%%%%%%%%%%%%%%%%%%%%%%%%%%%%%%%%%%%

%-----------------------------------------------------------------
% INCLUDING PACKAGES
%-----------------------------------------------------------------
\documentclass[11pt]{article}
\usepackage[a4paper,margin=2.4cm]{geometry}
\usepackage[T1]{fontenc}
\usepackage[utf8]{inputenc}
\usepackage[english,italian]{babel}
\usepackage{amsmath,amsthm}
\usepackage{booktabs,amsmath}
\usepackage{amsfonts}
\usepackage{feynmp-auto}
\usepackage{slashed}
\usepackage{titling,lipsum}
\usepackage{fancyhdr}
\usepackage{graphicx}
\usepackage{multicol}
\usepackage{longtable}
\usepackage{titlesec}
\usepackage[figurename=Fig.]{caption}
\usepackage[tablename=Tabella]{caption}
\usepackage[style=numeric,sorting=ynt]{biblatex}

% COURTESY OF BETOHAKU FROM STACKEXCHANGE
\newcommand{\marrow}[5]{%
    \fmfcmd{style_def marrow#1
    expr p = drawarrow subpath (1/4, 3/4) of p shifted 7 #2 withpen pencircle scaled 0.4;
    label.#3(btex #4 etex, point 0.5 of p shifted 9 #2);
    enddef;}
    \fmf{marrow#1,tension=0}{#5}}

\renewcommand{\thesection}{\Roman{section}} % use roman numerals for section numbering
\titleformat*{\section}{\large\bfseries}    % set section heading format

\addbibresource{books.bib}

%\pagestyle{fancy}
%\fancyhf{}
%\rhead{\slshape \leftmark}
\cfoot{\thepage} % page number
%-----------------------------------------------------------------

% \setlength{\columnseprule}{1pt}

\setlength{\droptitle}{-6em}   % This is your set screw
\begin{document}
\renewcommand{\abstractname}{\vspace{-\baselineskip}}

    %-----------------------------------------------------------------
    % TITLE
    %-----------------------------------------------------------------
    
    \begin{center} % TITOLO PROVVISORIO: Il processo $e^+ e^- \to \mu ^+ \mu ^-$ in approssimazione di Born nel Modello Standard
        \LARGE{\textbf{\textsc{Il processo $e^+ e^- \to \mu ^+ \mu ^-$ in approssimazione di Born nel Modello Standard}}}
        \bigskip\\\large{\textsc{Mario Ciacco}}
        \\\large{\textsc{Matricola 835681}}
    \end{center}
    \normalsize

    %-----------------------------------------------------------------
    % ABSTRACT
    %-----------------------------------------------------------------

    \begin{abstract}
    In questo \textit{essay} sono calcolati i termini d'interazione fermione-antifermione-scalare nel Modello Standard necessari per il calcolo
    completo dell'ampiezza del processo ed è verificata l'indipendenza dalla scelta di gauge dell'elemento di matrice $S$ all'ordine perturbativo più basso.
    $\mathrm{\grave{E}}$ infine presentato il calcolo della sezione d'urto differenziale.
    \end{abstract}

    \unitlength = 0.5mm % Feynman diagrams dimensions
    
    % -------------
    % INTRODUCTION
    % -------------
    \section{\centering\textsc{Introduzione} } % Cioè l'ultima sezione da scrivere
    Le interazioni elettrodeboli sono descritte in modo unificato nella lagrangiana con gruppo di simmetria $SU(2)\times U(1)$. In tale modello, i leptoni interagiscono con i quattro campi di gauge 
    $W^\pm_\mu$, $Z_\mu$ e $A_\mu$, e con il doppietto scalare complesso $\Phi$ necessario per l'introduzione dei termini di massa. Nel caso particolare del processo $e^+e^-\to\mu^+\mu^-$, all'ordine più basso in teoria delle perturbazioni, 
    la parte di lagrangiana d'interazione $\mathcal{L}_{int}$ da considerare è formata da due contributi separati. Il primo è il termine di correnti neutre
    % neutral current lagrangian for electron and muon
    \begin{equation}
    \mathcal{L}_{nc}=\sum_{i=e,\mu}\left\{\frac{g}{4\cos\theta_W}\bar{\ell}_i\gamma^\mu(4\sin^2\theta_W-1+\gamma^5)\ell_iZ_\mu-e\bar{\ell}_i\gamma^\mu\ell_iA_\mu\right\}
    \end{equation}
    da cui si estrae il vertice a tre linee con il campo $Z_\mu$ (risp. $A_\mu$) e la coppia leptone-antileptone, di valore $(2\pi)^4i(g/4c)\gamma^\mu(V+\gamma^5)$ (risp. $-(2\pi)^4ie\gamma^\mu$) nello spazio degli impulsi. Si è posto $c\equiv\cos\theta_W$ e $V\equiv4\sin^2\theta_W+1$.
    Il secondo contributo è dato dall'interazione dei campi dell'elettrone e del muone con le componenti del campo $\Phi$.

    % --------------------------------------
    % THE SCALAR FIELD AND ITS INTERACTIONS
    % --------------------------------------
    \section{\centering\textsc{L'interazione con il campo scalare} } 
    La lagrangiana del campo scalare è
    % Scalar field lagrangian
    \begin{equation}
    \mathcal{L}_s=\left(D_\mu\Phi\right)^\dagger D^\mu\Phi-\frac{\lambda}{4!}\left(\lvert\Phi\rvert^2-F^2\right)^2, \ \ \ \ \lambda>0\ \mathrm{e} \ F\neq0
    \end{equation}
    dove $D_\mu$ è la derivata covariante in termini dei campi di gauge e dei generatori del gruppo di simmetria. La costante $F$ determina il valore di aspettazione sul vuoto del campo $\Phi$, classicamente corrispondente
    al valore assunto da $\lvert\Phi\rvert$ nel minimo del potenziale. Si parametrizza $\Phi$ in termini di $F$ e se ne esprime la seconda componente attraverso una combinazione del campo fisico di Higgs $H$ e del campo $\phi^0$.
    Il fattore $2^{-\frac{1}{2}}$ permette di ottenere i corretti termini cinetici per i campi reali
    % Scalar field definition in terms of Higgs and phi0
    \begin{equation}\label{PhiParameterization}
    \Phi=\begin{pmatrix}0 \\F\end{pmatrix}+\begin{pmatrix}\phi_1 \\\phi_2\end{pmatrix}=\begin{pmatrix}0\\F\end{pmatrix}+\frac{1}{\sqrt{2}}\begin{pmatrix}\sqrt{2}\phi_1 \\H+i\phi^0\end{pmatrix}
    \end{equation}
    L'interazione con i campi fermionici è descritta attraverso gli accoppiamenti di Yukawa
    % Scalar-fermion lagrangian (only muon and electron are used)
    \begin{equation}\label{YukawaCouplings}
    \mathcal{L}_{sf}=-\sum_{i=e,\mu}\left\{y_i\bar{\psi}_{i,L}\Phi\psi_{i,R}+h.c.\right\}
    \end{equation}
    dove $y_j$ è la costante di accoppiamento di Yukawa, proporzionale al rapporto tra la massa $m_j$ del fermione e la massa $M_W$ dei bosoni $W^\pm$
    % Yukawa coupling constant
    \begin{equation}\label{YukawaConstant}
    y_j=\frac{1}{\sqrt{2}}g\frac{m_j}{M_W}
    \end{equation}
    I campi fermionici sono introdotti in (\ref{YukawaCouplings}) con $\bar{\psi}_{j,L}=2^{-1}\left(\bar{\nu}_j,\bar{\ell}_j\right)(1+\gamma^5)$ e $\psi_{j,R}=2^{-1}(1+\gamma^5)\ell_{j}$.
    Inserendo la parametrizzazione (\ref{PhiParameterization}) nella lagrangiana d'interazione (\ref{YukawaCouplings}), isolando la seconda componente del doppietto,
    si esplicitano i termini di lagrangiana d'interazione per i singoli campi.
    % Computation of scalar-lepton vertices
    \begin{equation}
    \begin{split}
    \mathcal{L}_{sf}\Longrightarrow\ \ -&\frac{y_j}{\sqrt{2}}\begin{pmatrix}\bar{\nu}_{j} & \bar{\ell}_{j}\end{pmatrix}\left(\frac{1+\gamma^5}{2}\right)\begin{pmatrix}-\\ \sqrt{2}F+H+i\phi^0\end{pmatrix}\left(\frac{1+\gamma^5}{2}\right)\ell_j\\
    -&\frac{y_j}{\sqrt{2}}\bar{\ell}_j\left(\frac{1-\gamma^5}{2}\right)\begin{pmatrix} \ - &\sqrt{2}F+H-i\phi^0\ \end{pmatrix}\left(\frac{1-\gamma^5}{2}\right)\begin{pmatrix} \nu_j\\ \ell_j \end{pmatrix}\\
    =-&y_j\bar{\ell}_jF\ell_j-\frac{y_j}{\sqrt{2}}\bar{\ell}_j H \ell_j-i\frac{y_j}{\sqrt{2}}\bar{\ell}_j \gamma^5\phi^0 \ell_j
    \end{split}
    \end{equation}
    Il primo termine è responsabile della massa dei fermioni. I termini d'interazione relativi rispettivamente a $H$ e $\phi^0$ sono dati dal prodotto di campi non equivalenti, pertanto i vertici corrispondenti si deducono in modo diretto dalla lagrangiana.
    Utilizzando l'espressione (\ref{YukawaConstant}) per la costante di accoppiamento, si ottengono rispettivamente i valori $-(2\pi)^4i[2^{-1}g(m_j/M_W)]$ e $(2\pi)^42^{-1}g(m_j/M_W)\gamma^5$ nello spazio degli impulsi.

    \section{\centering\textsc{L'indipendenza dell'elemento di matrice \textit{S} dalla scelta di gauge} } % verificata per il diagramma ad albero, quindi solo per un ordine perturbativo
    La teoria che si sta considerando è una teoria di gauge, pertanto la sua quantizzazione richiede l'introduzione di un termine \textit{gauge fixing} secondo il metodo di Faddeev-Popov. In analogia a quanto sviluppato per il modello di Higgs abeliano,
    è possibile introdurre una condizione dipendente da un parametro $\xi$ e definire così la famiglia di gauge $R_\xi$. In questo caso $\phi^0$ è il campo non fisico, presente nello spettro della teoria con $\xi$ di valore generico, associato al campo $Z_\mu$.
    In generale, quindi, data la lagrangiana d'interazione complessiva $\mathcal{L}_{int}=\mathcal{L}_{nc}+\mathcal{L}_{sf}$, si isolano i seguenti diagrammi di Feynman indipendenti all'ordine $O(g^2)$ ($\sim O(e^2)$, essendo $e=g\sin\theta_W$) dello sviluppo perturbativo
    dell'elemento di matrice $S$

    \smallskip
    \begin{fmffile}{amplFull}
        \begin{equation}\label{Diagrams}
        \begin{gathered}
        \begin{fmfgraph*}(60,30)
            \fmfleft{i1,i2}
            \fmfright{o1,o2}
            \fmflabel{$e^-$}{i1}
            \fmflabel{$e^+$}{i2}
            \fmflabel{$\mu^-$}{o1}
            \fmflabel{$\mu^+$}{o2}
            \fmf{photon,label=$\gamma$}{v1,v2}
            \fmf{fermion}{i1,v1}
            \fmf{fermion}{v1,i2}
            \fmf{fermion}{v2,o1}
            \fmf{fermion}{o2,v2}
        \end{fmfgraph*}
        \end{gathered}+
        \begin{gathered}
        \begin{fmfgraph*}(60,30)
            \fmfleft{i1,i2}
            \fmfright{o1,o2}
            \fmf{vanilla,label=$Z$}{v1,v2}
            \fmf{fermion}{i1,v1}
            \fmf{fermion}{v1,i2}
            \fmf{fermion}{v2,o1}
            \fmf{fermion}{o2,v2}
        \end{fmfgraph*}
        \end{gathered}+
        \begin{gathered}
        \begin{fmfgraph*}(60,30)
            \fmfleft{i1,i2}
            \fmfright{o1,o2}
            \fmf{dashes,label=$\phi^0$}{v1,v2}
            \fmf{fermion}{i1,v1}
            \fmf{fermion}{v1,i2}
            \fmf{fermion}{v2,o1}
            \fmf{fermion}{o2,v2}
        \end{fmfgraph*}
        \end{gathered}+
        \begin{gathered}
        \begin{fmfgraph*}(60,30)
            \fmfleft{i1,i2}
            \fmfright{o1,o2}
            \fmf{dashes,label=$H$}{v1,v2}
            \fmf{fermion}{i1,v1}
            \fmf{fermion}{v1,i2}
            \fmf{fermion}{v2,o1}
            \fmf{fermion}{o2,v2}
        \end{fmfgraph*}
        \end{gathered}
        \end{equation}
    \end{fmffile}

    \normalsize
    Per calcolare l'elemento di matrice $S$ si parte dal calcolo dell'ampiezza, in cui si media sullo spin dei fermioni entranti e si somma sullo spin dei fermioni uscenti
    % Full tree-level amplitude
    \begin{equation}
    \frac{1}{4}\sum_{spin}\lvert M\rvert^2=\frac{1}{4}\sum_{spin}\lvert M_1+ M_2+ M_3+ M_4\rvert^2
    \end{equation}
    dove gli indici degli elementi $M_i$ si riferiscono ai diagrammi precedenti nell'ordine con cui sono riportati.
    In analogia ai risultati del calcolo nel modello di Higgs abeliano, si possono esprimere i propagatori dei campi $Z_\mu$ e $\phi^0$ nello spazio degli impulsi come
    % Propagators in general R_xi gauge
    \begin{equation}
    iG_{F,Z}^{\mu\nu}(p)=\frac{-i}{(2\pi)^4}\frac{1}{p^2-M_Z^2}\left\{g^{\mu\nu}-(1-\xi^2)\frac{p^\mu p^\nu}{p^2-\xi^2M_Z^2}\right\};\ \ \ \ \ i\Delta_{F,\phi}(p)=\frac{i}{(2\pi)^4}\frac{1}{p^2-\xi^2M_Z^2}
    \end{equation}
    % inserire riferimento alla conservazione dell'impulso
    Data la dipendenza da $\xi$ dei propagatori, sia $M_2$ sia $M_3$ dipendono dal valore di $\xi$. Affinché la teoria sia consistente, si deve verificare l'indipendenza dell'elemento di matrice $S$ dalla scelta del gauge.
    Si considera quindi la somma di questi due termini e in particolare si isolano le parti dipendenti dal parametro $\xi$. Si trascurano da qui i fattori $(2\pi)^4$ presenti nei vertici e nei propagatori.
    % M2+M3 - first considerations -> only considering gauge dependent terms
    \begin{equation}\label{M2M3xiparts1}
    \begin{split}
     M_2+ M_3\Rightarrow -&\left(\frac{g}{4c}\right)^2 \frac{i(1-\xi^2)}{s^2-M_Z^2}\frac{(p_1+p_2)^\mu(p_3+p_4)^\nu}{s-\xi^2 M_Z^2}\bar{v}(p_1)\gamma_\mu(V+\gamma^5)u(p_2)\bar{u}(p_3)\gamma_\nu(V+\gamma^5)v(p_4)\\
    +&\left(\frac{g}{2}\right)^2\frac{m_e m_\mu}{M_W^2}\frac{i}{s-\xi^2M_Z^2}\bar{v}(p_1)\gamma^5u(p_2)\bar{u}(p_3)\gamma^5v(p_4)
    \end{split}
    \end{equation}
    In (\ref{M2M3xiparts1}) si sono associati i quadrimpulsi $p_i$ ($i=1,\dots,4$) alle linee esterne del diagramma e si è applicata la conservazione del quadrimpulso nei vertici interni del diagramma. Si è inoltre utilizzato l'invariante di Mandelstam $s=(p_1+p_2)^2$
    per esprimere il quadrato del quadrimpulso associato alle linee interne. Considerando esclusivamente il termine $M_2$ si utilizzano le equazioni di Dirac $\slashed{p}u(p)=mu(p)$ e $\slashed{p}v(p)=-mv(p)$ e quelle rispettive per gli spinori $\bar{u}(p)$ e $\bar{v}(p)$
    per eliminare i quadrimpulsi $p_i$ nell'espressione (\ref{M2M3xiparts1}) ed esplicitare la dipendenza dalle masse $m_e$ e $m_\mu$ rispettivamente dell'elettrone e del muone
    % M2 - Dirac equation applied
    \begin{equation}
    \begin{split}
    M_2\Rightarrow-\left(\frac{g}{4c}\right)^2 \frac{i(1-\xi^2)}{s^2-M_Z^2}\frac{1}{s-\xi^2 M_Z^2}\Big\{ \Big.&m_e[-\bar{v}(p_1)\gamma_\mu(V+\gamma^5)u(p_2)+\bar{v}(p_1)\gamma_\mu(V-\gamma^5)u(p_2)] \\
    \times &m_\mu[\bar{u}(p_3)\gamma_\nu(V+\gamma^5)v(p_4)-\bar{u}(p_3)\gamma_\nu(V-\gamma^5)v(p_4)]\Big. \Big\}
    \end{split}
    \end{equation}
    Si elimina la dipendenza dai termini di accoppiamento vettoriale e si mette in evidenza un fattore $m_em_\mu$ comune anche a $M_3$. In $M_3$ si utilizza la relazione $M_W=cM_Z$ che lega le masse dei bosoni $W^\pm$ e $Z$. La somma dei due contributi è infine
    esprimibile come
    % Gauge invariance proof
    \begin{equation}\label{proofGauge}
    \begin{split}
    M_2+M_3\Rightarrow\left(\frac{g}{4c}\right)^2\frac{4im_e m_\mu}{s-\xi^2M_Z^2}\left\{\frac{M_Z^2-\xi^2M_Z^2+s-M_Z^2}{M_Z^2(s-M_Z^2)}\right\}\bar{v}(p_1)\gamma^5u(p_2)\bar{u}(p_3)\gamma^5v(p_4)
    \end{split}
    \end{equation}
    Il numeratore in parentesi è uguale al denominatore ad esso precedente e questi sono gli unici termini dipendenti dal parametro $\xi$. Si verifica pertanto a quest'ordine perturbativo l'indipendenza dalla scelta di $\xi$. Si osserva inoltre che
    il termine ottenuto nell'espressione (\ref{proofGauge}) corrisponde all'elemento $M_3$ che si calcolerebbe con la scelta di gauge normalizzabile $\xi=1$ o alternativamente, riutilizzando le equazioni di Dirac, al termine di $M_2$ dovuto alla parte in
    $p^\mu p^\nu/M_Z^2$ del propagatore del campo $Z_\mu$ nel gauge unitario ($\xi\to\infty$).

    \section{\centering\textsc{La sezione d'urto differenziale} } % Contributi QED e Weak INSERIRE A QUESTO PUNTO LE CONSIDERAZIONI SULL'ACCOPPIAMENTO CON IL CAMPO SCALARE
    Si procede nel calcolo dell'elemento di matrice $S$ scegliendo $\xi=1$. Si riportano i valori di $M_j$ relativi ai diagrammi (\ref{Diagrams}). I propagatori del fotone e del campo di Higgs fisico $H$ sono rispettivamente
    $iG_{F,A}^{\mu\nu}(k)=-ig^{\mu\nu}/k^2$ e $i\Delta_{F,H}(p)=i/(p^2-M_H^2)$, data $M_H$ la massa del bosone di Higgs.
    \begin{align}
    M_1=&\left(-ie\right)^2\bar{v}(p_1)\gamma^\mu u(p_2)\frac{-ig_{\mu\nu}}{(p_1+p_2)^2}\bar{u}(p_3)\gamma^\nu v(p_4)\label{M1}\\
    M_2=&\left(\frac{ig}{4c}\right)^2\bar{v}(p_1)\gamma^\mu(V+\gamma^5) u(p_2)\frac{-ig_{\mu\nu}}{(p_1+p_2)^2-M_Z^2}\bar{u}(p_3)\gamma^\nu(V+\gamma^5) v(p_4)\label{M2}\\
    M_3=&\left(\frac{g}{2}\right)^2\frac{m_em_\mu}{M_W^2}\bar{v}(p_1)\gamma^5u(p_2)\frac{i}{(p_1+p_2)^2-M_Z^2}\bar{u}(p_3)\gamma^5v(p_4)\label{M3}\\
    M_4=&\left(\frac{-ig}{2}\right)^2\frac{m_em_\mu}{M_W^2}\bar{v}(p_1)u(p_2)\frac{i}{(p_1+p_2)^2-M_H^2}\bar{u}(p_3)v(p_4)\label{M4}
    \end{align}
    \subsection*{I termini di pura elettrodinamica e di pura interazione debole}
    Poiché si è interessati alla sezione d'urto non polarizzata, si somma sullo spin dei fermioni uscenti e si media sullo spin dei fermioni entranti la quantità $\lvert M_1\rvert^2$, ottenendo
    % QED: (spin averaged) modulus squared of amplitude
    \begin{equation}
    \begin{split}
    \frac{1}{4}\sum_{spin}\lvert M_1\rvert^2=\frac{1}{4}e^4\frac{1}{(p_1+p_2)^2}&\mathrm{tr}\{(\slashed{p}_2+m_e)\gamma^\nu(\slashed{p}_1-m_e)\gamma^\mu\} \\
     \times&\mathrm{tr}\{(\slashed{p}_4-m_\mu)\gamma_\nu(\slashed{p}_3+m_\mu)\gamma_\mu\}
    \end{split}
    \end{equation}
    Svolgendo l'algebra delle tracce di matrici di Dirac ed esprimendo il risultato in termini degli invarianti di Mandelstam $s$, $t$ e $u$ si ottiene
    % QED -> Mandelstam (spin averaged) modulus squared of amplitude
    \begin{equation}\label{QEDM1SQUARE}
    \frac{1}{4}\sum_{spin}\lvert M_1\rvert^2=\frac{2e^4}{s^2}\{(u-m_e^2-m_\mu^2)^2+(t-m_e^2-m_\mu^2)^2+2(m_e^2+m_\mu^2)s\}
    \end{equation}
    Analogamente si ottiene il termine $\lvert M_2\rvert^2$ relativo all'interazione debole
    % WEAK -> Mandelstam (spin averaged) modulus squared of amplitude
    \begin{equation}\label{WEAKM2SQUARE}
    \begin{split}
    \frac{1}{4}\sum_{spin}\lvert M_2\rvert^2=2\left(\frac{g}{4c}\right)^4\frac{1}{(s-M_Z^2)^2}\{&[(V^2+1)^2+4V^2](u-m_e^2-m_\mu^2)^2\\
    +&[(V^2+1)^2-4V^2](t-m_e^2-m_\mu^2)^2\\
    +&2(V^4-1)[(m_\mu^2+m_e^2)s-4m_\mu^2 m_e^2]\\
    +&8m_\mu^2 m_e^2(V^2-1)^2\}
    \end{split}
    \end{equation}
    In entrambi i termini sono presenti combinazioni degli invarianti $t$ e $u$ che determinano una dipendenza dagli angoli dei fermioni uscenti. 
    \subsection*{Il contributo dei campi scalari $\phi^0$ e $H$}
    % PHI0 -> Mandelstam (spin averaged) modulus squared of amplitude
    A partire dalle espressioni (\ref{M3}) e (\ref{M4}) si calcolano le ampiezze per i diagrammi relativi alle interazioni con i campi scalari
    \begin{equation}
    \begin{split}
    \frac{1}{4}\sum_{spin}\lvert M_3\rvert^2=\left(\frac{g}{2}\right)^4\left(\frac{m_em_\mu}{M_W^2}\right)^2\frac{1}{(s-M_Z^2)^2}\{&(s-2m_e^2)(s-2m_\mu^2)\\
     +&2(m_e^2+m_\mu^2)s-4m_e^2m_\mu^2\}
    \end{split}
    \end{equation}
    % HIGGS -> Mandelstam (spin averaged) modulus squared of amplitude
    \begin{equation}
    \begin{split}
    \frac{1}{4}\sum_{spin}\lvert M_4\rvert^2=\left(\frac{ig}{2}\right)^4\left(\frac{m_em_\mu}{M_W^2}\right)^2\frac{1}{(s-M_H^2)^2}\{&(s-2m_e^2)(s-2m_\mu^2)\\
     -&2(m_e^2+m_\mu^2)s+12m_e^2m_\mu^2\}
    \end{split}
    \end{equation}
    Differentemente da quanto si è verificato per i contributi (\ref{QEDM1SQUARE}) e (\ref{WEAKM2SQUARE}), i risultati ottenuti sono esprimibili in termini del solo invariante di Mandelstam $s$.
    La dipendenza angolare è in questo caso uniforme: questo fatto è riconducibile all'assenza di spin sulle linee interne dei diagrammi.
    \subsection*{I termini d'interferenza}
    Dovendo considerare 4 diagrammi indipendenti, si ottengono in principio $4!\cdot(2!)^{-1}=12$ termini d'interferenza. Si verifica che i termini relativi alla stessa coppia di diagrammi sono
    equivalenti. Per simmetria restano pertanto $6$ contributi indipendenti. Si considera per primo il termine d'interferenza dato da $M_3M_1^{\dagger}$ e nel calcolo non polarizzato si trascurano le costanti
    % QED x PHI0
    \begin{equation}
    \begin{split}
    \frac{1}{4}\sum_{spin} M_3 M_1^{\dagger}=\frac{1}{4}\sum_{spin} M_1 M_3^{\dagger} \Rightarrow \mathrm{tr}\{(\slashed{p}_2+m_e)\gamma^5(\slashed{p}_1-m_e)\gamma^{\mu}\}\mathrm{tr}\{(\slashed{p}_4-m_\mu)\gamma^5(\slashed{p}_3+m_\mu)\gamma_{\mu}\}
    \end{split}
    \end{equation}
    Espandendo le parentesi si ottengono esclusivamente tracce della forma $\mathrm{tr}\{\gamma^5\gamma^\mu\}$, $\mathrm{tr}\{\gamma^5\gamma^\mu\gamma^\nu\}$ e $\mathrm{tr}\{\gamma^5\gamma^\mu\gamma^\nu\gamma^\alpha\}$, tutte nulle.
    Analogamente si ottiene che il contributo $4^{-1}\sum_{spin}M_4M_3^{\dagger}$ sono nulli per considerazioni analoghe sulle tracce di matrici di Dirac
    % HIGGS x PHI0
    \begin{equation}
    \begin{split}
    \frac{1}{4}\sum_{spin} M_3 M_4^{\dagger}=\frac{1}{4}\sum_{spin} M_4 M_3^{\dagger} \Rightarrow \mathrm{tr}\{(\slashed{p}_2+m_e)\gamma^5(\slashed{p}_1-m_e)\}\mathrm{tr}\{(\slashed{p}_4-m_\mu)\gamma^5(\slashed{p}_3+m_\mu)\}=0
    \end{split}
    \end{equation}
    Si riportano di seguito i 4 contributi indipendenti non nulli restanti
    % WEAK x QED -> Mandelstam (spin averaged) modulus squared of amplitude
    \begin{equation}\label{WEAKxQEDinterf}
    \begin{split}
    \frac{1}{4}\sum_{spin} M_2 M_1^{\dagger}=\frac{1}{4}\sum_{spin} M_1 M_2^{\dagger}=2e^2\left(\frac{g}{4c}\right)^2\frac{1}{s(s-M_Z^2)}\{&(V^2+1)(u-m_e^2-m_\mu^2)^2\\
    +&(V^2-1)(t-m_e^2-m_\mu^2)^2\\
    +&2V^2(m_e^2+m_\mu^2)s\}
    \end{split}
    \end{equation}
    % HIGGS x QED
    \begin{equation}
    \frac{1}{4}\sum_{spin} M_4 M_1^{\dagger}=\frac{1}{4}\sum_{spin} M_1 M_4^{\dagger}=-\left(\frac{g}{2}\right)^2\frac{(m_em_\mu)^2}{M_W^2}\frac{e^2}{s(s-M_H^2)}\{ (u-m_e^2-m_\mu^2)-(t-m_e^2-m_\mu^2) \}
    \end{equation}
    % PHI0 x WEAK
    \begin{equation}
    \frac{1}{4}\sum_{spin} M_3 M_2^{\dagger}=\frac{1}{4}\sum_{spin} M_2 M_3^{\dagger}=-\left(\frac{g}{2}\right)^2\frac{(m_em_\mu)^2}{M_W^2}\left(\frac{g}{4c}\right)^2\frac{1}{(s-M_Z^2)^2}\{ (u-m_e^2-m_\mu^2)+(t-m_e^2-m_\mu^2) \}
    \end{equation}
    % HIGGS x WEAK
    \begin{equation}
    \frac{1}{4}\sum_{spin} M_4 M_2^{\dagger}=\frac{1}{4}\sum_{spin} M_2 M_4^{\dagger}=-\left(\frac{g}{2}\right)^2\frac{(m_em_\mu)^2}{M_W^2}\left(\frac{g}{4c}\right)^2\frac{1}{(s-M_Z^2)(s-M_H^2)}\{ (u-m_e^2-m_\mu^2)-(t-m_e^2-m_\mu^2) \}
    \end{equation}
    \subsection*{Cinematica e risultati finali}
    Per passare alla sezione d'urto, si moltiplica l'ampiezza calcolata per il fattore di flusso $F$ e per l'integrale sullo spazio delle fasi delle particelle uscenti dal processo. Si considera il sistema di riferimento del centro di massa del sistema.
    % flux factor and phase space integral
    \begin{equation}
	\begin{split}
	\dfrac{d\sigma}{dt} = \left(2\pi\right)^{4-6}F \int \mathrm{d}\underset{2\to2}{PS}\ \dfrac{1}{4}\sum_{spin}\lvert M\rvert^2 = \dfrac{1}{\left(2\pi\right)^{2}} \cdot \dfrac{1}{2s^{\frac{1}{2}}(s-4m_e^2)^{\frac{1}{2}} } \cdot \dfrac{\pi}{2s^{\frac{1}{2}}(s-4m_e^2)^{\frac{1}{2}} } \cdot \dfrac{1}{4}\sum_{spin}\ \sum_{i,j=1}^{4} M_i M_j^{\dagger}
	\end{split}
	\end{equation}
    Si passa alla sezione d'urto differenziale rispetto all'angolo $\theta$ che l'antimuone uscente forma rispetto all'asse del moto dell'elettrone esprimendo gli invarianti $t=(m_\mu^2+m_e^2+s/2-2^{-1}(s-4m_\mu^2)^{\frac{1}{2}}(s-4m_e^2)^{\frac{1}{2}}\cos\theta)$
    e $u=(m_\mu^2+m_e^2-s/2-2^{-1}(s-4m_\mu^2)^{\frac{1}{2}}(s-4m_e^2)^{\frac{1}{2}}\cos\theta)$. Si applica la trasformazione $d\sigma/d\cos\theta=2^{-1}(s-4m_e^2)^{\frac{1}{2}}(s-4m_\mu^2)^{\frac{1}{2}}d\sigma/dt$
    % THE cross section
    \begin{equation}
    \resizebox{0.9\textwidth}{!}{
        $\begin{split}
    &\frac{d\sigma}{d\cos\theta}=\frac{1}{32\pi s}\left(\frac{s-4m_\mu^2}{s-4m_e^2}\right)^{\frac{1}{2}} \Bigg\{ \Bigg. \frac{e^4}{s^2}\Big\{ \Big. s^2+(s-4m_\mu^2)(s-4m_e^2)\cos^2\theta+4(m_e^2+m_\mu^2)s\Big. \Big\} \\
    +&\left(\frac{g}{4c}\right)^4\frac{1}{(s-M_Z^2)^2}\Big\{ \Big. (V^2+1)^2\left[s^2+(s-4m_e^2)(s-4m_\mu^2)\cos^2\theta\right]+8V^2s(s-4m_e^2)^{\frac{1}{2}}(s-4m_\mu^2)^{\frac{1}{2}}\cos\theta\\
    +&4(V^4-1)\left[(m_\mu^2+m_e^2)s-4m_\mu^2 m_e^2\right]+16m_\mu^2 m_e^2(V^2-1)^2 \Big. \Big\}\\
    +&2e^2\left(\frac{g}{4c}\right)^2\frac{1}{s(s-M_Z^2)}\Big\{ \Big. V^2\left[ s^2+(s-4m_e^2)(s-4m_\mu^2)\cos^2\theta \right]+2s(s-4m_e^2)^{\frac{1}{2}}(s-4m_\mu^2)^{\frac{1}{2}}\cos\theta+4V^2(m_e^2+m_\mu^2) \Big. \Big\}
    \Bigg. \Bigg\} 
        \end{split}$}
    \end{equation}
    Il fattore $(s-4m_\mu^2)^{\frac{1}{2}}$ garantisce che la sezione d'urto sia definita per $\sqrt{s}\leq2m_\mu$, cioè solo quando l'energia nel sistema di riferimento del centro di massa è sufficiente per creare la massa della coppia $\mu^+\mu^-$.
    I termini in $\cos\theta$ sono responsabili dell'asimmetria della sezione d'urto rispetto all'angolo di diffusione $\theta$.
    \nocite{sterman}
    \nocite{veltman}
    %\nocite{bjorkendrell}
    %\nocite{peskinschroeder}
    %\nocite{pdg}
    %\nocite{appunti}
    \medskip

    \printbibliography

\end{document}