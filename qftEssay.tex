
%%%%%%%%%%%%%%%%%%%%%%%%%%%%%%%%%%%%%%%%%%%%%%%%%%%%%%%%%%%%%%%%%%
%%                 DOCUMENT FOR QFT 0 ESSAY                     %%
%%%%%%%%%%%%%%%%%%%%%%%%%%%%%%%%%%%%%%%%%%%%%%%%%%%%%%%%%%%%%%%%%%

%-----------------------------------------------------------------
% INCLUDING PACKAGES
%-----------------------------------------------------------------
\documentclass[11pt]{article}
\usepackage[a4paper,margin=2.5cm]{geometry}
\usepackage[T1]{fontenc}
\usepackage[utf8]{inputenc}
\usepackage[english,italian]{babel}
\usepackage{amsmath,amsthm}
\usepackage{booktabs,amsmath}
\usepackage{amsfonts}
\usepackage{feynmp-auto}
\usepackage{slashed}
\usepackage{titling,lipsum}
\usepackage{fancyhdr}
\usepackage{graphicx}
\usepackage{multicol}
\usepackage{longtable}
\usepackage{titlesec}
\usepackage[figurename=Fig.]{caption}
\usepackage[tablename=Tabella]{caption}
\usepackage[style=numeric,sorting=ynt]{biblatex}

% COURTESY OF BETOHAKU FROM STACKEXCHANGE
\newcommand{\marrow}[5]{%
    \fmfcmd{style_def marrow#1
    expr p = drawarrow subpath (1/4, 3/4) of p shifted 7 #2 withpen pencircle scaled 0.4;
    label.#3(btex #4 etex, point 0.5 of p shifted 9 #2);
    enddef;}
    \fmf{marrow#1,tension=0}{#5}}

\renewcommand{\thesection}{\Roman{section}} % use roman numerals for section numbering
\titleformat*{\section}{\large\bfseries}    % set section heading format

\addbibresource{books.bib}

%\pagestyle{fancy}
%\fancyhf{}
%\rhead{\slshape \leftmark}
\cfoot{\thepage} % page number
%-----------------------------------------------------------------

% \setlength{\columnseprule}{1pt}

\setlength{\droptitle}{-8em}   % This is your set screw
\begin{document}
\renewcommand{\abstractname}{\vspace{-\baselineskip}}

    %-----------------------------------------------------------------
    % TITLE
    %-----------------------------------------------------------------
    
    \begin{center} % TITOLO PROVVISORIO: Il processo $e^+ e^- \to \mu ^+ \mu ^-$ in approssimazione di Born nel Modello Standard
        \LARGE{\textbf{\textsc{Il processo $e^+ e^- \to \mu ^+ \mu ^-$ in approssimazione di Born nel Modello Standard}}}
        \bigskip\\\large{\textsc{Mario Ciacco}}
        \\\large{\textsc{Matricola 835681}}
    \end{center}
    \normalsize

    %-----------------------------------------------------------------
    % ABSTRACT
    %-----------------------------------------------------------------

    \begin{abstract}
    ABSTRACT
    \end{abstract}

    \unitlength = 0.5mm % Feynman diagrams dimensions
    
    % -----------------
    % FEYNMAN DIAGRAMS
    % -----------------

    \section{\centering\textsc{Introduzione} } % Cioè l'ultima sezione da scrivere

    % Full tree-level amplitude
    \begin{equation}
    \lvert\mathcal{M}\rvert^2=\lvert\mathcal{M}_1+\mathcal{M}_2+\mathcal{M}_3+\mathcal{M}_4\rvert^2
    \end{equation}


    \section{\centering\textsc{L'interazione con il campo scalare} }

    % Scalar field lagrangian
    \begin{equation}
    \mathcal{L}_s=\left(D_\mu\Phi\right)^\dagger D^\mu\Phi+\frac{\lambda}{4!}\left(\lvert\Phi\rvert^2+F^2\right)^2, \ \ \ \ \lambda>0
    \end{equation}

    % Scalar field definition in terms of Higgs and phi0
    \begin{equation}
    \Phi=\begin{pmatrix}0 \\F\end{pmatrix}+\begin{pmatrix}\phi_1 \\\phi_2\end{pmatrix}=\begin{pmatrix}0\\F\end{pmatrix}+\frac{1}{\sqrt{2}}\begin{pmatrix}\sqrt{2}\phi_1 \\H+i\phi^0\end{pmatrix}
    \end{equation}

    % Scalar-fermion lagrangian (only muon and electron are used)
    \begin{equation}
    \mathcal{L}_{sf}=\sum_{i=e,\mu}y_i\bar{\psi}_{i,L}\Phi\psi_{i,R}+h.c.
    \end{equation}

    % Yukawa coupling constant
    \begin{equation}
    y_j=\frac{1}{\sqrt{2}}g\frac{m_j}{M_W}
    \end{equation}

    % Computation of scalar-lepton vertices
    \begin{equation}
    \begin{split}
    \mathcal{L}_{sf}\Longrightarrow\ &\frac{y_j}{\sqrt{2}}\begin{pmatrix}\bar{\nu}_{j} & \bar{\ell}_{j}\end{pmatrix}\left(\frac{1+\gamma^5}{2}\right)\begin{pmatrix}-\\ \sqrt{2}F+H+i\phi^0\end{pmatrix}\left(\frac{1+\gamma^5}{2}\right)\ell_j+h.c.\\
    =&y_j\bar{\ell}_jF\ell_j+\frac{y_j}{\sqrt{2}}\bar{\ell}_j H \ell_j+i\frac{y_j}{\sqrt{2}}\bar{\ell}_j \gamma^5\phi^0 \ell_j
    \end{split}
    \end{equation}

    % muon-to-W mass ratio
    \begin{equation}
    \frac{m_{\mu}}{M_W}\simeq 10^{-3}
    \end{equation}


    \section{\centering\textsc{La sezione d'urto differenziale} } % Contributi QED e Weak

    % neutral current lagrangian for electron and muon
    \begin{equation}
    \mathcal{L}_{nc}=\sum_{i=e,\mu}\left\{\frac{g}{4\cos\theta_W}\bar{\ell}_i\gamma^\mu(4\sin^2\theta_W-1+\gamma^5)\ell_iZ_\mu-e\bar{\ell}_i\gamma^\mu\ell_iA_\mu\right\}
    \end{equation}

    % QED contribution
    \begin{fmffile}{gamma}
        \begin{equation*}
        \begin{gathered}
        \begin{fmfgraph*}(60,30)
            \fmfleft{i1,i2}
            \fmfright{o1,o2}
            \fmflabel{$e^-$}{i1}
            \fmflabel{$e^+$}{i2}
            \fmflabel{$\mu^-$}{o1}
            \fmflabel{$\mu^+$}{o2}
            \fmf{photon,label=$\gamma$}{v1,v2}
            \fmf{fermion}{i1,v1}
            \fmf{fermion}{v1,i2}
            \fmf{fermion}{v2,o1}
            \fmf{fermion}{o2,v2}
            % Momentum arrows
            \marrow{a}{left}{lft}{$p_1$}{i1,v1}
            \marrow{b}{left}{lft}{$p_2$}{i2,v1}
            \marrow{c}{up}{top}{$p_1+p_2$}{v1,v2}
            \marrow{d}{right}{rt}{$p_3$}{v2,o1}
            \marrow{e}{right}{rt}{$p_4$}{v2,o2}
        \end{fmfgraph*}
        \end{gathered} \ \ \ \Longrightarrow\mathcal{M}_1=\left(-ie\right)^2\bar{v}(p_1)\gamma^\mu u(p_2)\frac{-ig_{\mu\nu}}{(p_1+p_2)^2}\bar{u}(p_3)\gamma^\nu v(p_4)
        \end{equation*}
    \end{fmffile}

    % QED: (spin averaged) modulus squared of amplitude
    \begin{equation}
    \begin{split}
    \frac{1}{4}\sum_{Spin}\lvert\mathcal{M}_1\rvert^2=\frac{1}{4}e^4\frac{1}{(p_1+p_2)^2}&\mathrm{tr}\{(\slashed{p}_2+m_e)\gamma^\nu(\slashed{p}_1-m_e)\gamma^\mu\} \\
     \times&\mathrm{tr}\{(\slashed{p}_4-m_\mu)\gamma_\nu(\slashed{p}_3+m_\mu)\gamma_\mu\}
    \end{split}
    \end{equation}

    % QED -> Mandelstam (spin averaged) modulus squared of amplitude
    \begin{equation}
    \frac{1}{4}\sum_{Spin}\lvert\mathcal{M}_1\rvert^2=\frac{2e^4}{s^2}\{(u-m_e^2-m_\mu^2)^2+(t-m_e^2-m_\mu^2)^2+2(m_e^2+m_\mu^2)s\}
    \end{equation}

    % Z-boson contribution (WEAK)
    \begin{fmffile}{zee}
        \begin{equation*}
        \begin{gathered}
        \begin{fmfgraph*}(60,30)
            \fmfleft{i1,i2}
            \fmfright{o1,o2}
            \fmflabel{$e^-$}{i1}
            \fmflabel{$e^+$}{i2}
            \fmflabel{$\mu^-$}{o1}
            \fmflabel{$\mu^+$}{o2}
            \fmf{vanilla,label=$Z$}{v1,v2}
            \fmf{fermion}{i1,v1}
            \fmf{fermion}{v1,i2}
            \fmf{fermion}{v2,o1}
            \fmf{fermion}{o2,v2}
            % Momentum arrows
            \marrow{f}{left}{lft}{$p_1$}{i1,v1}
            \marrow{g}{left}{lft}{$p_2$}{i2,v1}
            \marrow{h}{up}{top}{$p_1+p_2$}{v1,v2}
            \marrow{i}{right}{rt}{$p_3$}{v2,o1}
            \marrow{l}{right}{rt}{$p_4$}{v2,o2}
        \end{fmfgraph*}
        \end{gathered} \ \ \ \Longrightarrow\mathcal{M}_2=\left(\frac{ig}{4c}\right)^2\bar{v}(p_1)\gamma^\mu(V+\gamma^5) u(p_2)\frac{-ig_{\mu\nu}}{(p_1+p_2)^2-M_Z^2}\bar{u}(p_3)\gamma^\nu(V+\gamma^5) v(p_4)
        \end{equation*}
    \end{fmffile}

    % WEAK -> Mandelstam (spin averaged) modulus squared of amplitude
    \begin{equation}
    \begin{split}
    \frac{1}{4}\sum_{Spin}\lvert\mathcal{M}_2\rvert^2=2\left(\frac{g}{4c}\right)^4\frac{1}{(s-M_Z^2)^2}\{&[(V^2+1)^2+4V^2](u-m_e^2-m_\mu^2)^2\\
    +&[(V^2+1)^2-4V^2](t-m_e^2-m_\mu^2)^2\\
    +&2(V^4-1)[(m_\mu^2+m_e^2)s-4m_\mu^2 m_e^2]\\
    +&8m_\mu^2 m_e^2(V^2-1)^2\}
    \end{split}
    \end{equation}

    % WEAK x QED -> Mandelstam (spin averaged) modulus squared of amplitude
    \begin{equation}
    \begin{split}
    \frac{1}{4}\sum_{Spin}\mathcal{M}_1\mathcal{M}_2^{\dagger}=2e^2\left(\frac{g}{4c}\right)^2\frac{1}{s(s-M_Z^2)}\{&(V^2+1)(u-m_e^2-m_\mu^2)^2\\
    +&(V^2-1)(t-m_e^2-m_\mu^2)^2\\
    +&2V^2(m_e^2+m_\mu^2)s\}
    \end{split}
    \end{equation}

    % CINEMATICA: LE DEFINIZIONI DEGLI IMPULSI INLINE

    % flux factor
    \begin{equation}
	\begin{split}
	\dfrac{d\sigma}{dt} & = \left(2\pi\right)^{4-6}\cdot F \cdot \int \mathrm{d}\underset{2\to2}{PS} \cdot \dfrac{1}{4}\sum_{Spin}\lvert\mathcal{M}\rvert^2 \\ 
	 & = \dfrac{1}{\left(2\pi\right)^{2}} \cdot \dfrac{1}{2\sqrt{s(s-4m_e^2)}} \cdot \dfrac{\pi}{2\sqrt{s(s-4m_\mu^2)}} \cdot \dfrac{1}{4}\sum_{Spin}\left\{\lvert\mathcal{M}_1\rvert^2+\lvert\mathcal{M}_2\rvert^2+2\mathcal{M}_1\mathcal{M}_2^{\dagger}\right\}
	\end{split}
	\end{equation}


    \section{\centering\textsc{L'ampiezza del diagramma con linea interna di Higgs} } % Caratteristiche di interazione con scalare

    % Higgs-boson contribution
    \begin{fmffile}{higgs}
        \begin{equation*}
        \begin{gathered}
        \begin{fmfgraph*}(60,30)
            \fmfleft{i1,i2}
            \fmfright{o1,o2}
            \fmflabel{$e^-$}{i1}
            \fmflabel{$e^+$}{i2}
            \fmflabel{$\mu^-$}{o1}
            \fmflabel{$\mu^+$}{o2}
            \fmf{dashes,label=$H$}{v1,v2}
            \fmf{fermion}{i1,v1}
            \fmf{fermion}{v1,i2}
            \fmf{fermion}{v2,o1}
            \fmf{fermion}{o2,v2}
            % Momentum arrows
            \marrow{m}{left}{lft}{$p_1$}{i1,v1}
            \marrow{n}{left}{lft}{$p_2$}{i2,v1}
            \marrow{o}{up}{top}{$p_1+p_2$}{v1,v2}
            \marrow{p}{right}{rt}{$p_3$}{v2,o1}
            \marrow{q}{right}{rt}{$p_4$}{v2,o2}
        \end{fmfgraph*}
        \end{gathered} \ \ \ \Longrightarrow\mathcal{M}_H=\left(\frac{ig}{2}\right)^2\frac{m_em_\mu}{M_W^2}\bar{v}(p_1)u(p_2)\frac{-i}{(p_1+p_2)^2-M_H^2}\bar{u}(p_3)v(p_4)
        \end{equation*}
    \end{fmffile}

    % HIGGS -> Mandelstam (spin averaged) modulus squared of amplitude
    \begin{equation}
    \begin{split}
    \frac{1}{4}\sum_{Spin}\lvert\mathcal{M}_H\rvert^2=\left(\frac{ig}{2}\right)^4\left(\frac{m_em_\mu}{M_W^2}\right)^2\frac{1}{(s-M_H^2)^2}\{&(s-2m_e^2)(s-2m_\mu^2)\\
     -&2(m_e^2+m_\mu^2)s+8m_e^2m_\mu^2\}
    \end{split}
    \end{equation}

    \nocite{sterman}
    \nocite{veltman}
    %\nocite{bjorkendrell}
    %\nocite{peskinschroeder}
    \nocite{pdg}
    \medskip

    \printbibliography

\end{document}